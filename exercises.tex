% Notes and exercises from Finite Dimensional Multilinear Algebra by Marcus
% By John Peloquin
\documentclass[letterpaper,12pt]{article}
\usepackage{amsmath,amssymb,amsthm,enumitem,fourier,diagrams}
\usepackage[hidelinks]{hyperref}

\DeclareMathOperator{\im}{Im}

\newcommand{\tprod}{\otimes}
\newcommand{\bigtprod}{\bigotimes}
\newcommand{\medtprod}{{\textstyle\bigtprod}}

\newcommand{\dual}[1]{#1^*}
\newcommand{\proj}[1]{\overline{#1}}
\newcommand{\multi}[4]{#2_{#3}#1\cdots#1#2_{#4}}
\newcommand{\timess}[3]{\multi{\times}{#1}{#2}{#3}}
\newcommand{\tprods}[3]{\multi{\tprod}{#1}{#2}{#3}}

% Theorems
\theoremstyle{definition}
\newtheorem*{exer}{Exercise}

\theoremstyle{remark}
\newtheorem*{rmk}{Remark}

% Meta
\title{Notes and exercises from\\\textit{Finite Dimensional Multilinear Algebra}}
\author{John Peloquin}
\date{}

\begin{document}
\maketitle

% Intro
\section*{Introduction}
This document contains notes and exercises from \cite{marcus1} and~\cite{marcus2}.

\bigskip
\noindent
{\boldmath\textbf{Unless otherwise stated, \(R\)~denotes a field of characteristic~\(0\) over which all vector spaces are defined.}}

% Part I
\newpage
\part*{Part~I}
\section*{Chapter~1}
\subsection*{\S~3}
\begin{exer}[9,10]
Let \(\varphi\in M(V_1,\ldots,V_m:U)\) in this commutative diagram:
\begin{diagram}[nohug]
\timess{V}{1}{m}&\rTo^{\tprod}	&\tprods{V}{1}{m}\\
				&\rdTo<{\varphi}&\dTo>f\\
				&				&U
\end{diagram}
Let \(K=\ker f\). Then \(\im\varphi=U\) if and only if \(f\)~is surjective and every element in \((\tprods{V}{1}{m})/K\) has a decomposable representative.
\end{exer}
\begin{proof}
If \(\im\varphi=U\), then \(f\)~is surjective and there is an induced isomorphism \(\proj{f}:(\tprods{V}{1}{m})/K\to U\). Also for any \(\proj{z}\in(\tprods{V}{1}{m})/K\) there are \(v_i\in V_i\) with
\[\proj{f}(\proj{\tprods{v}{1}{m}})=f(\tprods{v}{1}{m})=\varphi(v_1,\ldots,v_m)=f(z)=\proj{f}(\proj{z})\]
Since \(\proj{f}\)~is injective, it follows that \(\proj{\tprods{v}{1}{m}}=\proj{z}\).

For the converse, if \(u\in U\) there are \(v_i\in V_i\) with \(\proj{\tprods{v}{1}{m}}=\proj{f}^{-1}(u)\), so
\[u=\proj{f}(\proj{\tprods{v}{1}{m}})=f(\tprods{v}{1}{m})=\varphi(v_1,\ldots,v_m)\]
Therefore \(\im\varphi=U\).
\end{proof}

\begin{rmk}
In Exercise~11, it is simpler to prove that \(\dual{(\tprods{V}{1}{m})}\) is a tensor product of \(\dual{V_1},\ldots,\dual{V_m}\), then use Theorem~2.4.
\end{rmk}

\begin{rmk}
In Exercise~12, it is simpler to observe that the linear map induced by~\(\nu\) through a tensor product is surjective between spaces of the same finite dimension and hence an isomorphism, then use Exercise~3.
\end{rmk}

\newpage
\section*{Chapter~2}
\subsection*{\S~1}
\begin{rmk}
The tensor product of linear maps of finite dimensional vector spaces is really a tensor product. More specifically, the map
\[\tprod:L(V_1,U_1)\times\cdots\times L(V_m,U_m)\to L(\tprods{V}{1}{m},\tprods{U}{1}{m})\]
defined by
\[(T_1,\ldots,T_m)\mapsto\tprods{T}{1}{m}\]
is a tensor product map, so
\[L(V_1,U_1)\tprod\cdots\tprod L(V_m,U_m)=L(\tprods{V}{1}{m},\tprods{U}{1}{m})\]
This is the content of Theorem~2.7, which should be in this section. Similarly the Kronecker product of matrices is a tensor product.
\end{rmk}

\begin{rmk}
In Example~1.2(b), it is simpler to prove~(13) directly from definition~(10) of the Kronecker product.
\end{rmk}

\subsection*{\S~2}
\begin{rmk}
In Exercise~4, it is also possible to use Example~2.6(c).
\end{rmk}

\subsection*{\S~4}
\begin{rmk}
In Exercise~11, it is simpler to use Exercise~10, together with Exercise~15 in Section~2.3.
\end{rmk}

\newpage
\section*{Chapter~3}
\subsection*{\S~1}
\begin{rmk}
It follows from Theorem~1.7, together with Theorem~3.4 in Section~2.3, that if \(\varphi\in M_m(V,R,S_m,1)\) and \(\varphi(x,\ldots,x)=0\) for all \(x\in V\), then \(\varphi=0\). In other words, \emph{a (completely) symmetric multilinear function is completely determined by its output values on equal input values}.
\end{rmk}

% References
\newpage
\begin{thebibliography}{0}
\bibitem{marcus1} Marcus, M. \textit{Finite Dimensional Multilinear Algebra~I}. 1973.
\bibitem{marcus2} Marcus, M. \textit{Finite Dimensional Multilinear Algebra~II}. 1975.
\end{thebibliography}
\end{document}
